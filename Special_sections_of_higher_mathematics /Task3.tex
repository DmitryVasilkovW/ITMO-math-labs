\documentclass{article}
\usepackage[russian]{babel}
\usepackage{graphicx} % Required for inserting images

\title{СРВМ ДЗ2}
\author{Dmitry Vasilkov}
\date{May 2023}

\begin{document}

\maketitle

\begin{center}
    \text{Вариант 15}
\end{center}

\begin{center}
    {
    \text{Задание 189}\\
    }
\end{center}


\begin{center}
    {$
    (1 - x^2y)dx + x^2(y - x)dy = 0,   \mu = \phi(x)\\
    $}
    
    {
    $P = 1 - x^2y\\$
    }

    {
    $Q = x^2(y - x)\\$
    }
    
    \text
    {
    Поскольку $\frac{1}{Q}(\frac{\delta P}{\delta y} - \frac{\delta Q}{\delta x}) = g(x)$, 
    $\mu (x) = e^{\int g(x)dx}\\$
    }

    {
    $\frac{1}{(x^2(y - x))}(\frac{\delta (1 - x^2y)}{\delta y} - \frac{\delta (x^2(y - x))}{\delta x}) = \\$
    }
    
    {$\frac{1}{(x^2(y - x))}((\frac{\delta (1)}{\delta y} + \frac{\delta (- x^2y)}{\delta y}) - (\frac{\delta (yx^2)}{\delta x} - \frac{\delta (x^3)}{\delta x})) = \frac{1}{(x^2(y - x))}(-x^2 - 2xy + 3x^2) = -\frac{2}{x}\\$
    }\\
    {
    $
    \mu (x) = e^{\int - \frac{2}{x}dx}\\
    $
    }

    {
    $
    \mu (x) = e^{-2ln|x| + c}\\
    $
    }

    {
    $
    \mu (x) = \frac{1}{x^2}\\
    $
    }

    {
    $
    (1 - x^2y)\frac{1}{x^2}dx + (x^2(y - x))\frac{1}{x^2}dy = 0\\
    $
    }

    {
    $
    \frac{\delta P}{\delta y} * \frac{1}{x^2} = \frac{\delta Q}{\delta x} * \frac{1}{x^2} = -1 \\
    $
    }\\
    {
    $
    M(x, y) = (1-x^2y)\frac{1}{x^2}\\$
    \\
    $N(x, y) = (x^2(y - x))\frac{1}{x^2}\\
    $
    }\\
    {
    $\frac{\delta \upsilon}{\delta x}(x, y) = M(x, y)$,
    }
    {
    $\frac{\delta \upsilon}{\delta y}(x, y) = N(x, y)\\$\\
    }\\
    {
    $
    \upsilon(x, y) = \int{(1 - x^2y)* \frac{1}{x^2}}dx\\
    $
    }\\
    {
    $
    \upsilon(x, y) = \int{\frac{(1 - x^2y)}{x^2}}dx\\
    $
    }\\
    {
    $
    \upsilon(x, y) = \int{\frac{1}{x^2} - y}dx\\
    $
    }\\
    {
    $
    \upsilon(x, y) = \int{\frac{1}{x^2}}dx - \int {y}dx\\
    $
    }\\
    {
    $
    \upsilon(x, y) = -\frac{1}{x} - xy + c(y)\\
    $
    }\\
    {
    $
    \frac{\delta \upsilon}{\delta y}(x, y) = -x + \frac{\delta}{\delta y}(c(y))\\
    $
    }\\
    {
    $
    \frac{\delta \upsilon}{\delta y}(x, y) = -x + \frac{\delta}{\delta y}(c(y))\\
    $
    }\\
    {
    $
    \frac{\delta \upsilon}{\delta y}(x, y) = (x^2 * (y - x))*\frac{1}{x^2}\\
    $
    }\\
    {
    $
    -x + \frac{\delta}{\delta y} (c(y)) = (x^2(y - x))* \frac{1}{x^2}\\
    $
    }\\
    {
    $
    -x + \frac{\delta}{\delta y} (c(y)) = (x^2y - x^3)* \frac{1}{x^2}\\
    $
    }\\
    {
    $
    -x + \frac{\delta}{\delta y} (c(y)) = x^2(y - x)* \frac{1}{x^2}\\
    $
    }\\
    {
    $
    -x + \frac{\delta}{\delta y} (c(y)) = y - x\\
    $
    }\\
    {
    $
    \frac{\delta}{\delta y} (c(y)) = y\\
    $
    }\\
    {
    $
    c(y) = \frac{y^2}{2} + c, c \in R\\
    $
    }\\
    {
    $
    \upsilon(x, y) = -\frac{1}{x} - xy + \frac{y^2}{2} + c, c \in R\\
    $
    }\\
    {
    $
    -\frac{1}{x} - xy + \frac{y^2}{2} + c = D, c \in R, D \in R\\
    $
    }\\
    {
    $
    -\frac{1}{x} - xy + \frac{y^2}{2} = c, c \in R\\
    $
    }\\
    {
    $
    -2 - 2x^2y + xy^2 = cx\\
    $
    }\\\\
\end{center}\\

\begin{center}
    {
    \text{Задание 289}\\
    }
\end{center}

\begin{center}
    {
    $y' + \cos(\frac{x + y}{2}) = \cos(\frac{x - y}{2})\\$
    }\\
    {
    $\frac{dy}{dx} + \cos(\frac{x + y}{2}) = \cos(\frac{x - y}{2})\\$
    }\\
    {
    $\frac{dy}{dx} = - \cos(\frac{x + y}{2}) + \cos(\frac{x - y}{2})\\$
    }\\
    {
    $\frac{dy}{dx} = - \cos(\frac{x}{2})\cos(\frac{y}{2}) - \sin(\frac{x}{2})\sin(\frac{y}{2}) + \cos(\frac{x}{2})\cos(\frac{y}{2}) - \sin(\frac{x}{2})\sin(\frac{y}{2})\\$
    }\\
    {
    $\frac{dy}{dx} = 2\sin(\frac{x}{2})\sin(\frac{y}{2})\\$
    }\\
    {
    $dy = 2\sin(\frac{x}{2})\sin(\frac{y}{2})dx\\$
    }\\
    {
    $\frac{dy}{\sin(\frac{y}{2})} = 2\sin(\frac{x}{2})dx\\$
    }\\
    {
    $M(y)dy = N(x)dx\\$
    }\\
    {
    $\int{\frac{1}{\sin(\frac{y}{2})}}dy = \int{2\sin(\frac{x}{2})}dx\\$
    }\\
    {
    $-2\ln(\frac{\cos(\frac{y}{2}) + 1}{\sin(\frac{y}{2})}) = c - 4 \cos(\frac{x}{2})\\$
    }\\
    {
    $\ln(\frac{\cos(\frac{y}{2}) + 1}{\sin(\frac{y}{2})}) = - \frac{c - 4\cos(\frac{x}{2})}{2}\\$
    }\\
\end{center}

\end{document}
