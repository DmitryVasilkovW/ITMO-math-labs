\documentclass{article}
\usepackage{graphicx}
\usepackage{tikz}
\usepackage[russian]{babel}

\title{Домащнее задание СРВМ}
\author{Выполнил: \textbf{Васильков Дмитрий M3115}}
\date{16 Марта 2023}

\begin{document}

\maketitle

\section{Геометрическая задачка}
\par{Найти кривую, проходящую через точку (2, 3) и обладающую тем свойством, что отрезок произвольной ее касательной, концы которого лежат на осях координат, делится точкой касания пополам.}\\

\par{Отметим точки A и B как точки, где кривая пересекает оси координат. Согласно условию задачи, точка касания делит отрезок AB на две равные части. Это означает, что абсцисса $x_{A}$ точки A вдвое больше, чем абсцисса $x_{0}$ по абсолютной величине. Чтобы составить дифференциальное уравнение кривой, нам необходимо определить значение $x_{A}$. Для этого мы можем записать уравнение касательной.}
$$y = y'(x_{0})(x-x_{0})+y_{0}$$
\par{$]y=0$}
$$0 = y'(x_{0})(x-x_{0})+y_{0}$$
\par{Выразим $x$:}
$$x=x_{0} - \frac{y_{0}}{y'(x_{0})}$$
\par{Заметим, что значение $x$ в точке A должно быть больше в два раза(из условия):}
$$x_{0} - \frac{y_{0}}{y'(x_{0})} = 2x_{0}$$
$$y'(x_{0}) = -x_{0} \cdot y_{0}$$
$$y' = -xy$$
\par{Решим наше уравнение:}
$$\frac{dy}{dx} = -xy$$
$$dy = -xy \cdot dx$$
$$\frac{dy}{y} = -x \cdot dx$$
$$\int{\frac{dy}{y}} = \int{-xdx}$$
$$ln|y| = -\frac{x^2}{2} + C$$
$$|y| = e^{-\frac{x^2}{2} + C}$$
$$y = e^{-\frac{x^2}{2} + C}$$
$$y = \frac{C}{e^{\frac{x^2}{2}}}$$
$$3 = \frac{C}{e^{\frac{2^2}{2}}}$$
$$C = 3e^2$$
\textbf{Формула кривой: $y = \frac{3e^2}{e^{\frac{x^2}{2}}}$}

\section{Разделяющиеся переменные}

\textbf{Задание №1}
$$\frac{dy}{dx} = \sin(x-y)$$
\textbf{Решение:}
$$dy = -\sin(y - x)dx$$
\text{Пусть $u = y - x$, тогда:
$$du = dy - dx$$
$$y = x + u$$
$$dy = dx + du$$}
$$dx + du = -sin(u)dx$$
$$du = (-\sin(u) - 1)dx$$
$$-\frac{du}{sin(u) + 1} = dx$$
$$\int-\frac{1}{\sin(u)+1}du = \int1dx$$
$$\tg({\frac{u}{2} - \frac{\pi}{4}}) = -x + C$$
\textbf{Ответ: $\tg({\frac{u}{2} - \frac{\pi}{4}}) = -x + C$}\\\\

\textbf{Задание №2}
$$e^{y'} = x$$
\textbf{Решение:}
$$e^{y'} = x$$
$$y' = \ln(x)$$
$$\frac{dy}{du} = \ln(x)$$
$$dy = \ln(x)dx$$
$$\int1dy = \int ln(x)dx$$
$$y = x \cdot \ln(x) - x + C$$
\textbf{Ответ: $y = x \cdot \ln(x) - x + C$}\\\\

\textbf{Задание №3}
\par{Найти частное решение в уравнении $x^2y'cosy + 1 = 0$, удовлетворяющее условию $\lim_{x\to\infty}{y} = \frac{16\pi}{3}$}\\
\textbf{Решение:}
$$y'\cos(y) = -\frac{1}{x^2}$$
$$\frac{dy}{dx}cosy = -\frac{1}{x^2}$$
$$\int \cos(y)dy = \int\frac{dx}{x^2}$$
$$\sin(y) = \frac{1}{x} + C$$
$$y = \arcsin(\frac{1}{x} + C)$$
$$\frac{16\pi}{3} = \arcsin(C)$$
$$c = \sin(\frac{16\pi}{3}) - \frac{\sqrt{3}}{2}$$
$$y = \arcsin(\frac{1}{x} - \frac{\sqrt{3}}{2})$$
\textbf{Ответ: $y = \arcsin(\frac{1}{x} - \frac{\sqrt{3}}{2})$}

\section{Однородные уравнения}
\textbf{Задание №1}
$$x(y-x)dy = y^{2}dx$$\\
\textbf{Решение:}\\\\
\text{]$x(y-x)$ = P(x, y), $y^2$ = Q(x, y)}\\\\
$$P(\lambda x, \lambda y) = \lambda x (\lambda y - \lambda x) = \lambda^2 (xy - x^2) = \lambda^2 P(x, y)$$
$$Q(\lambda x, \lambda y) = (\lambda y)^2 = \lambda^2y^2 = \lambda^2 Q(x, y)$$
$$x(y - x)dy = y^2dx$$
$$x(y-x)y' = y^2$$
$$(xy - x^2)y' = y^2$$
$$y' = \frac{y^2}{xy - x^2}$$
\text{]$y = ux$, $u = \frac{y}{x}$, $y' = u'x + u$: 
$$u'x + u = \frac{u^2}{u - 1}$$
$$\frac{du}{dx}x + u = \frac{u^2}{u - 1}$$
$$\frac{(u - 1)du}{u} = \frac{dx}{x}$$
$$\int\frac{u-1}{u}du = \int\frac{1}{x}dx$$
$$u - ln|u| = ln|x| + C$$}
$$\frac{y}{x} - ln(\frac{y}{x}) = lnx + C$$
\textbf{Ответ: $\frac{y}{x} - ln(\frac{y}{x}) = lnx + C$}\\

\textbf{Задание №2}
$$2x^2y' = x^2 + y^2$$\\
\textbf{Решение:}\\\\
$$y' = \frac{x^2 + y^2}{2x^2}$$
$$y' = \frac{1}{2} + \frac{y^2}{2x^2}$$
\text{]$y = ux$, $u = \frac{y}{x}$, $y' = u'x + u$:
$$u'x + u = \frac{1}{2} + \frac{1}{2}u^2$$
$$u'x = \frac{u^2 - 2u + 1}{2}$$
$$u' = \frac{u^2 - 2u + 1}{2x}$$
$$\frac{du}{(u^2 - 2u + 1)dx} = \frac{1}{2x}$$
$$\frac{du}{u^2 - 2u + 1} = \frac{dx}{2x}$$
$$\int\frac{du}{u^2 - 2u + 1} = \int\frac{dx}{2x}$$
$$\int\frac{du}{(u- 1)^2} = \frac{1}{2} \int\frac{dx}{x}$$
$$\frac{du}{t^2} = \frac{ln|x|}{2} + C$$
$$-\frac{1}{t} = \frac{ln|x|}{2} + C$$
$$-\frac{1}{u - 1} = \frac{ln|x|}{2} + C$$
$$-\frac{1}{\frac{y}{x} - 1} = \frac{lnx}{2} + C$$}\\\\
\textbf{Ответ: $-\frac{1}{\frac{y}{x} - 1} = \frac{lnx}{2} + C$}\\\\

\textbf{Задание №3}
$$xy' = y(\ln(y) - \ln(x))$$\\
\textbf{Решение:}\\\\
$$y' = \frac{y(\ln(y) - \ln(x))}{x}$$
\text{]$y = ux$, $u = \frac{y}{x}$, $y' = u'x + u$:
$$y' = \frac{y}{\ln(\frac{y}{x})}$$
$$u'x + u = \frac{u}{\frac{1}{u}}$$
$$u'x = u^2 + u$$
$$u' = \frac{u^2 + u}{x}$$
$$\frac{du}{dx} = \frac{u^2 + u}{x}$$
$$\frac{du}{u^2 + u} = \frac{dx}{x}$$
$$\int\frac{du}{u^2 + u} = \frac{dx}{x}$$
$$\int\frac{1}{u} - \frac{1}{u+1}du = \ln|x| + C$$
$$\int\frac{1}{u}du - \int\frac{1}{u + 1}du = ln|x| + C$$
$$\ln|u| - \ln|u + 1| = \ln|x| + C$$
$$\ln(\frac{y}{x}) - \ln(\frac{y}{x} + 1) = \ln(x) + C$$
$$\ln(\frac{y}{y + x}) = \ln(x) + C$$}\\\\
\textbf{Ответ: $\ln(\frac{y}{y + x}) = \ln(x) + C$}

\end{document}
